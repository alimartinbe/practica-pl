% Clase
\documentclass[11pt,a4paper,spanish,twoside]{book}

% Órdenes auxiliares
% Español
\usepackage[spanish]{babel}
\usepackage[utf8]{inputenc}

\usepackage{lmodern}
\usepackage[T1]{fontenc}
%\usepackage{textcomp}
%\usepackage{times}

% Símbolo del euro
\usepackage{textcomp}

%Profundidad de numeración
\setcounter{secnumdepth}{3}

% Imágenes
\usepackage[pdftex]{graphicx}
\usepackage{latexsym}
\usepackage{fancybox}

% Tablas
\usepackage{array}

% Enumeraciones
\usepackage{enumerate}

% Ruta para las imágenes
\graphicspath{{img/}}

% Rotaciones
\usepackage[twoside]{rotating}

% Multirow
\usepackage{multirow}

% Referencias
\usepackage[spanish]{varioref}
\usepackage[pdftex,colorlinks=true,linkcolor=black]{hyperref}

% Colores
\usepackage{color}
\usepackage{colortbl}

% Párrafos
\setlength{\parskip}{6pt}

% Code for creating empty pages
% No headers on empty pages before new chapter
\makeatletter
\def\cleardoublepage{\clearpage\if@twoside \ifodd\c@page\else
    \hbox{}
    \thispagestyle{empty}
    \newpage
    \if@twocolumn\hbox{}\newpage\fi\fi\fi}
\makeatother \clearpage{\pagestyle{empty}\cleardoublepage}

%Para las secciones y las subseciones de las actas
\makeatletter
\renewcommand{\section}{%
  \@startsection{section}{1}{0mm}{\baselineskip}%
  {0.1mm}{\large\bf}%
}%
\renewcommand{\subsection}{%
  \@startsection{subsection}{2}{0mm}{2mm}%
  {0.1mm}{\normalsize\bf}%
}%
\renewcommand{\subsubsection}{%
  \@startsection{subsubsection}{3}{0mm}{2mm}%
  {0.1mm}{\normalsize\bf\emph}%
}%
\makeatother
\usepackage{titlesec}

\titleformat{\chapter}[display]
  {\bfseries\Large}
  {\filleft\MakeUppercase{\chaptertitlename} \Huge\thechapter}
  {4ex}
  {\titlerule
   \vspace{2ex}%
   \filright}
  [\vspace{2ex}%
   \titlerule]

% Fancyhdr - Encabezados y pies de página
\usepackage{fancyhdr}
% Márgenes
\headsep=8mm
\footskip=14mm

% Fancy Header Style Options
\pagestyle{fancy}               % Sets fancy header and footer
\fancyfoot{}                    % Delete current footer settings

% Sin mayúsculas en la cabecera
\lhead{\nouppercase{\rightmark}}
\rhead{\nouppercase{\leftmark}}

% Capítulo
\renewcommand{\chaptermark}[1]{ % Lower Case Chapter marker style
  \markboth{\chaptername\ \thechapter.\ #1}{}} 

% Sección
\renewcommand{\sectionmark}[1]{ % Lower case Section marker style
  \markright{\thesection.\ #1}} 

% Página
\fancyhead[LE,RO]{\bfseries\thepage} % Page number (boldface) in left on even
                                     % pages and right on odd pages

% ------------------ Macro para encabezados y pies de página-------------------
%    Uso: \encabezado{pares(pag izquierda)}
% -----------------------------------------------------------------------------
\def\encabezado{
  \fancyhead[RE]{\bfseries\leftmark}     % In the right on even pages
  \fancyhead[LO]{\bfseries\rightmark}  % In the left on odd pages
  \renewcommand{\headrulewidth}{0.5pt} % Width of head rule
}
% -----------------------------------------------------------------------------

% ------------------ Macro para insertar una imagen ---------------------------
%    Uso: \imagen{nombreFichero}{Ancho(cm))}{Etiqueta}{Identificador}
% -----------------------------------------------------------------------------
\usepackage{float}
\usepackage{ifthen}
\def\imagen#1#2#3#4{
  \begin{figure}[H]
    \begin{center}
      \ifthenelse{\equal{#2}{}}
      {\includegraphics{#1}}{\resizebox{#2cm}{!}{\includegraphics{#1}}}
      \ifthenelse{\equal{#3}{}}{}{\caption{#3}}
      \label{#4}
    \end{center}
  \end{figure}
}
% -----------------------------------------------------------------------------


% ------------------ Macro para la portada ------------------------------------
%    Uso: \portada{asignatura}{titulo}{subtítulo}{autor}{fecha}
% -----------------------------------------------------------------------------
\def\portada#1#2#3#4#5{
  \thispagestyle{empty}
  \vspace*{-3.3cm}

  \begin{minipage}[t]{14cm}
    \begin{center}
      \includegraphics[scale=0.25]{logo}
  
      \vspace*{1.5cm}
      {\Large \textbf{Universidad de Castilla-La Mancha\\ 
          Escuela Superior de Informática}\\}
    
      \vspace*{1.2cm}
      {\huge \textbf{#1}\\}
    
      \vspace*{1.5cm}
      {\huge #2\\}{\Large #3\\}
    
      \vspace*{1.5cm}
      {\large #4\\}
      \vspace*{1.4cm}
      \large{#5}
    \end{center}
  \end{minipage}

  \newpage
%   \vspace*{1cm}
%   \thispagestyle{empty} 
%   \newpage
}
% -----------------------------------------------------------------------------


% ------------------ Macro para la licencia -----------------------------------
%    Uso: \portada{asignatura}{titulo}{subtítulo}{autor}{fecha}
% -----------------------------------------------------------------------------
\def\licencia#1{
  \thispagestyle{empty}  % Suprime la numeración de esta página
  \vspace*{16.5cm}
  \begin{small}
    \copyright~ #1. Se permite la copia, distribución y/o 
    modificación de este documento bajo los términos de la licencia de
    documentación libre GNU, versión 1.1 o cualquier versión posterior publicada
    por la {\em Free Software Foundation}, sin secciones invariantes. Puede
    consultar esta licencia en http://www.gnu.org. \\[0.2cm]
    Este documento fue compuesto con \LaTeX{}. 
  \end{small}
  \newpage
%   \thispagestyle{empty}
%   \vspace*{0cm}
%   \newpage
}
% -----------------------------------------------------------------------------




\usepackage{amsthm}
\usepackage{verbatim}
\usepackage{multirow}
\usepackage{rotating}

%Para las secciones y las subseciones
\makeatletter
\renewcommand{\section}{
  \@startsection{section}{1}{0mm}{\baselineskip}
  {2.5mm}{\Large\bf}
}
\renewcommand{\subsection}{
  \@startsection{subsection}{2}{0mm}{2mm}
  {4.0mm}{\bf}
}
\renewcommand{\subsubsection}{
  \@startsection{subsubsection}{3}{0mm}{2mm}
  {0.1mm}{\normalsize\bf\emph}
}
\makeatother

\theoremstyle{plain} \newtheorem{nota}{Nota}

% Encabezado y pie de página
\encabezado

\begin{document}

% Portada
\portada{Procesadores de Lenguaje}
{2ª Entrega}{Resumen actual de la práctica\\\emph{Generador de mini-aventuras 
gráficas en 2D}}{Alicia Martín-Benito Escalona\\Juan Miguel Torres Triviño}{14 
de Febrero de 2012}  

% Licencia
\licencia{Alicia Martín-Benito Escalona, Juan Miguel Torres Triviño} 

% Índices
\tableofcontents
\listoftables
\listoffigures
\chapter{Introducción del problema}
El objetivo del lenguaje a definir es un generador de mini-aventuras gráficas 
en 2D, la primera idea es definir un escenario con un personaje, una serie de 
objetos y una serie de acciones que si no se realizan en dicho orden, no se 
cumple el objetivo de la mini-aventura, hacer feliz al personaje. La idea surgió
de un juego sencillo que tiene bastantes versiones en internet, llamado 
\emph{Monkey Go Happy}, este juego da una serie de objetos y realizándolos en 
orden consigue hacer sonreír al mono. Un ejemplo de escenario de este juego se 
puede ver en la imagen \ref{mono}.
\imagen{img/monkey-go-happy-44.jpg}{10}{Ejemplo Monkey Go Happy.}{mono}

Ya que dicho problema permite cualquier tipo de situación, por tanto definimos 
una serie de fondos, personajes, ubicaciones, objetos y acciones. De esta forma
,elegimos un fondo, un personaje, una serie de objetos, la ubicación de los 
objetos y el personaje y por último, una serie de acciones sobre los objetos 
donde la última acción es hacer al personaje feliz. Además, definimos un 
estado inicial, donde se ubican personajes y objetos en el escenario o fondo.

Después de definir toda la mini-aventura en 2D, el usuario dándole la situación
inicial, debe pulsar el ratón al igual que en el juego de ejemplo y así cumplir
todas las acciones en orden para finalizar la mini-aventura.
 
\section{Diagrama en T}

El diseño del sistema queda representado por el Diagrama en T que muestra la
figura \ref{diagrama}.
\imagen{img/DiagramT.jpg}{11}{Diagrama en T del sistema.}{diagrama}

\section{Lenguaje de entrada}
Un ejemplo del lenguaje de entrada es el siguiente:
\begin{verbatim}
#Hay un mono en una habitación que tiene 6 posiciones espaciales: pa, pb, 
pc, pd, arriba y abajo. Hay 3 objetos: llave, puerta y caja. El mono está 
en la posición pa, la llave está colgada ( arriba) en la posición pc, la 
caja está en el suelo(abajo) en la posición pb y la puerta está en la 
posición pd. El mono debe subirse a la caja para conseguir la llave y abrir 
la puerta, con esto estará feliz.#

escenario{ fondo2, mono, (caja, puerta, llave)}

inicio{ mono(pa,abajo), caja(pb,abajo), llave(pc,arriba), puerta(pd,abajo)}

posacciones{ mono(ir, feliz), caja(mover, subira, bajarde), llave(coger,usar), 
puerta(abrir)}

plan{ ir(pa,pb), mover(caja,pb,pc),subira(caja,pc,arriba), coger(llave,pc), 
bajarde(caja,pc,abajo), ir(pc,pd), usar(llave,pd), abrir(puerta,pd), 
feliz(mono)}
\end{verbatim}

\chapter{Notación EBNF}
La notación \emph{EBNF} de la práctica conforme a la estructura anterior queda 
de la siguiente manera:
\\ 
\\
JUEGO ::= ESCENARIO INICIAL RESTRICCION PLAN
\\
DESCRIPCION ::= \# \{LETRA | NUM | ESPACIO | OTRAS\} \#
\\
ESCENARIO ::= \textbf{escenario} '\{' FONDO ',' PERSONAJE ',' ( OBJETO { ',' 
OBJETO} ) '\}'
\\
FONDO ::= \textbf{fondo1} | \textbf{fondo2} | \textbf{fondo3} | \textbf{fondo4}
\\
PERSONAJE ::= \textbf{mono} | \textbf{pajaro} | \textbf{gato} 
\\
OBJETO ::= \textbf{llave} | \textbf{caja} | \textbf{puerta} | \textbf{bombilla}
 | \textbf{maceta} | \textbf{jarron} | \textbf{tele} | \textbf{rama} | 
\textbf{fruta} | \textbf{agua} | \textbf{semilla} 
\\
INICIAL ::= \textbf{inicio} '\{' PERSONAJE (POSICION ',' UBICACION) ',' OBJETO 
(POSICION ',' UBICACION) \{ ',' OBJETO (POSICION ',' UBICACION) \} '\}'
\\
POSICION ::= '\textbf{pa}' | '\textbf{pb}' | '\textbf{pc}' | '\textbf{pd}'
\\
UBICACION ::= '\textbf{arriba}' | '\textbf{abajo}'
\\
RESTRICCION ::= \textbf{posacciones} '\{' PERSONAJE (ACCION \{',' ACCION\})',' 
OBJETO (ACCION \{',' ACCION\}) \{',' OBJETO (ACCION {',' ACCION\})\}
\\
PLAN ::= \textbf{plan} '\{' ACCIONES \{ ',' ACCIONES \}'\}'
\\ 
ACCIONES ::= ACCION (PERSONAJE) | ACCION (POSICION, POSICION) | 
ACCION (OBJETO, POSICION) | ACCION (OBJETO, POSICION, POSICION) | 
ACCION (OBJETO, POSICION, UBICACION)
\\
ACCION ::= \textbf{mover} | \textbf{coger} | \textbf{abrir} | \textbf{subira} | 
\textbf{echar} | \textbf{encender} | \textbf{usar} | \textbf{romper} | 
\textbf{comer} | \textbf{apagar} | \textbf{salir} | \textbf{bajarde} | 
\textbf{ir} | \textbf{feliz}
\\
LETRA ::= MAYUS | MINUS
\\
MAYUS ::= A|B|C|D|E|F|G|H|I|J|K|L|M|N|Ñ|O|P|Q|R|S|T|U|V|W|X|Y|Z
\\
MINUS ::= a|b|c|d|e|f|g|h|i|j|k|l|m|n|ñ|o|p|q|r|s|t|u|v|w|x|y|z
\\
NUM ::= 0|1|2|3|4|5|6|7|8|9|
\\
ESPACIO ::= \#x20 | $\backslash$t | $\backslash$n | $\backslash$r
\\
OTRAS ::= ,|(|)|\{|\}

\chapter{Diagramas de Conway}
Algunas de las construcciones sintácticas definidas en el capítulo anterior
pueden ser complicadas entender haciendo uso solamente de la notación EBNF,
por lo cual en este capítulo describimos dichas construcciones mediante
diagramas de Conway.

\begin{itemize}
\item ESCENARIO.
\imagen{img/escenario.png}{15}{Diagrama de Conway para Escenario.}{}
\item INICIAL.
\imagen{img/inicial.png}{15}{Diagrama de Conway para Inicial.}{}
\item PLAN.
\imagen{img/plan.png}{15}{Diagrama de Conway para Plan.}{}
\item RESTRICCION.
\imagen{img/restriccion.png}{15}{Diagrama de Conway para Restriccion} {}
\item ACCIONES.
\imagen{img/acciones.png}{15}{Diagrama de Conway para Acciones}{}
\end{itemize}

\chapter{Primeras restricciones semánticas del lenguaje}
Las restriciones semánticas iniciales que hemos encontrado son las siguientes:
\begin{itemize}
\item En la definición de los objetos del escenario, ya que no tenemos límite
de objetos, debemos tener en cuenta que no se repitan objetos.
\item En la definición de \emph{inicio}, donde situamos los objetos y el 
personaje, no pueden aparecer objetos o personajes que no hayan sido definidos 
en el escenario.
\item Deben de aparecer todos los objetos y el personaje en la parte del código 
de \emph{inicio}, previamente definidos todos en la parte \emph{escenario}.
\item No puede haber dos objetos en una misma posición, sin embargo, pueden 
aparecer el personaje y un objeto en la misma posición.
\item Las restricciones definidas en \emph{posacciones} serán de libre elección
por el usuario, por lo cual, debe saber que acciones se realizan sobre qué 
objeto o personaje.
\item En \emph{posacciones} deben aparecer los objetos y el personaje que hayan
sido definidos en \emph{escenario}. Por el contrario no deben aparecer objetos
no definidos.
\item En la parte \emph{plan} deben aparecer los objetos y el personaje que 
hayan sido definidos en \emph{escenario}. No deben de aparecer objetos no 
definidos.
\item En la parte \emph{plan} sólo puede haber acciones que hayan sido elegidas
para los objetos o el personaje en la parte \emph{posacciones}.
\item Las acciones con argumentos definidas en el \emph{plan} se deben comprobar
que pueden ser factibles para esas definiciones previas, siempre dando libertad
para crear cualquier tipo de situación.
\item Debe aparecer el objetivo final de hacer feliz al personaje.
\end{itemize}

\chapter{Tokens}
\section{Tabla descriptiva}
En el cuadro \ref{tokens} se pueden ver los tokens de la grámatica sobre el 
generador de miniaventuras gráficas en 2D. En dicha tabla se indica el nombre 
del token, posibles valores, si tiene atributos y la expresión regular 
correspondiente.

\begin{sidewaystable}[!ht]
  \centering
  \begin{tabular}{|c|c|c|c|}
    \hline
     \textbf{NOMBRE} & \textbf{VALORES} & \textbf{ATRIBUTOS} & 
     \textbf{EXPRESIÓN REGULAR}\\\hline\hline
     \emph{Escenario} & escenario & palabra reservada & escenario\\\hline
     \emph{Inisec} & \{ & & \{\\\hline
     \emph{Finsec} & \} & & \}\\\hline
     \emph{Coma} & , & & ,\\\hline
     \emph{Parizq} & ( & & (\\\hline
     \emph{Parder} & ) & & )\\\hline
     \emph{Fondo} & fondo1,fondo2... & lexema & fondo1|fondo2|fondo3|fondo4\\
     \hline
     \emph{Personaje} & mono,pajaro... & lexema & mono|pajaro|gato\\\hline
     & & & llave|caja|puerta|bombilla|\\
     \emph{Objeto} & puerta,llave... & lexema & maceta|jarron|tele|rama|\\
     & & & fruta|agua|semilla\\\hline
     \emph{Inicio} & inicio & palabra reservada & inicio\\\hline
     \emph{Posición} & pa,pb... & lexema & pa|pb|pc|pd\\\hline
     \emph{Ubicación} & arriba,abajo & lexema & arriba|abajo\\\hline
     \emph{Posacciones} & posacciones & palabra reservada & posacciones\\\hline
     \emph{Plan} & plan & palabra reservada & plan\\\hline
     & & & mover|coger|abrir|subira|echar|\\
     \emph{Acción} & mover,coger... & lexema & ir|encender|usar|romper|
     bajarde|\\
     & & & feliz|comer|apagar|salir\\\hline
  \end{tabular}
  \caption{Tabla de tokens}\label{tokens}
\end{sidewaystable}

\section{Autómatas Finitos Determinitas}
En esta sección se procede a mostrar los autómatas finitos de cada una de las 
expresiones regulares del cuadro \ref{tokens}. Los autómatas son los siguientes:

\begin{itemize}
\item Autómata del token \emph{Escenario}, figura \ref{afesc}.
\item Autómata del token \emph{Inisec}, figura \ref{afis}.
\item Autómata del token \emph{Finsec}, figura \ref{affs}.
\item Autómata del token \emph{Separador}, figura \ref{afs}.
\item Autómata del token \emph{Parizq}, figura \ref{afpi}.
\item Autómata del token \emph{Parder}, figura \ref{afpd}.
\item Autómata del token \emph{Fondo}, figura \ref{aff}.
\item Autómata del token \emph{Personaje}, figura \ref{afp}.
\item Autómata del token \emph{Objeto}, figura \ref{afo}.
\item Autómata del token \emph{Inicio}, figura \ref{afini}.
\item Autómata del token \emph{Posición}, figura \ref{afpos}.
\item Autómata del token \emph{Ubicación}, figura \ref{afu}.
\item Autómata del token \emph{Posacciones}, figura \ref{afpa}.
\item Autómata del token \emph{Plan}, figura \ref{afpl}
\item Autómata del token \emph{Acción}, figura \ref{afa}.
\end{itemize}

\imagen{img/AutomataEscenario.png}{8}{Autómata del token escenario.}{afesc}
\imagen{img/AutomataInicioSeccion.png}{8}{Autómata del token inisec.}{afis}
\imagen{img/AutomataFinSeccion.png}{8}{Autómata del token finsec.}{affs}
\imagen{img/AutomataSeparador.png}{8}{Autómata del token separador.}{afs}
\imagen{img/AutomataParIz.png}{8}{Autómata del token parizq.}{afpi}
\imagen{img/AutomataParDer.png}{8}{Autómata del token parder.}{afpd}
\imagen{img/AutomataFondo.png}{8}{Autómata del token fondo.}{aff}
\imagen{img/AutomataPersonaje.png}{8}{Autómata del token personaje.}{afp}
\imagen{img/AutomataObjetos.png}{8}{Autómata del token objeto.}{afo}
\imagen{img/AutomataInicio.png}{8}{Autómata del token inicio.}{afini}
\imagen{img/AutomataPosicion.png}{8}{Autómata del token posición.}{afpos}
\imagen{img/AutomataUbicacion.png}{8}{Autómata del token ubicación.}{afu}
\imagen{img/AutomataPosacciones.png}{8}{Autómata del token posacciones.}{afpa}
\imagen{img/AutomataPlan.png}{8}{Autómata del token plan.}{afpl}
\imagen{img/AutomataAcciones.png}{8}{Autómata del token acción.}{afa}

\chapter{Gramática EBNF corregida}

La notación \emph{EBNF} de la práctica final, es decir, ya está corregida, 
queda de la siguiente manera:
\\ 
\\
JUEGO ::= ESCENARIO INICIAL RESTRICCION PLAN
\\
\\
ESCENARIO ::= \textbf{Escenario} \textbf{Inisec} \textbf{Fondo} \textbf{Coma} 
\textbf{Personaje} \textbf{Coma} \textbf{Parizq} \textbf{Objeto} \{ 
\textbf{Coma} \textbf{Objeto} \} \textbf{Parder} \textbf{Finsec}
\\
\\
INICIAL ::= \textbf{Inicio} \textbf{Inisec} \textbf{Personaje} \textbf{Parizq}
\textbf{Posicion} \textbf{Coma} \textbf{Ubicacion} \textbf{Parder} 
\textbf{Coma} \textbf{Objeto} \textbf{Parizq} \textbf{Posicion} \textbf{Coma} 
\textbf{Ubicacion} \textbf{Parder} \{ \textbf{Coma} \textbf{Objeto} 
\textbf{Parizq} \textbf{Posicion} \textbf{Coma} \textbf{Ubicacion} 
\textbf{Parder} \} \textbf{Finsec}
\\
\\
RESTRICCION ::= \textbf{Posacciones} \textbf{Inisec} \textbf{Personaje} 
\textbf{Parizq} \textbf{Accion} \{ \textbf{Coma} \textbf{Accion} \} 
\textbf{Parder} \textbf{Coma} \textbf{Objeto} \textbf{Parizq} \textbf{Accion} 
\{ \textbf{Coma} \textbf{Accion} \} \textbf{Parder} \{ \textbf{Coma} 
\textbf{Objeto}  \textbf{Accion} \{ \textbf{Coma} \textbf{Accion} \} 
\textbf{Parder} \textbf{Finsec}
\\
\\
PLAN ::= \textbf{Plan} \textbf{Inisec} ACCIONES \{ \textbf{Coma} ACCIONES \}
\textbf{Finsec}
\\
\\ 
ACCIONES ::= \textbf{Accion} \textbf{Parizq} \textbf{Personaje} \textbf{Parder}
| \\\textbf{Accion} \textbf{Parizq} \textbf{Posicion} \textbf{Coma} 
\textbf{Posicion} \textbf{Parder} |\\ \textbf{Accion} \textbf{Parizq} 
\textbf{Objeto} \textbf{Coma} \textbf{Posicion} \textbf{Parder} |\\ 
\textbf{Accion} \textbf{Parizq} \textbf{Objeto} \textbf{Coma} \textbf{Posicion}
\textbf{Coma} \textbf{Posicion} \textbf{Parder} |\\ \textbf{Accion} 
\textbf{Parizq} \textbf{Objeto} \textbf{Coma} \textbf{Posicion} \textbf{Coma} 
\textbf{Ubicacion} \textbf{Parder}

\end{document}
